\documentclass{beamer}
\usetheme{Madrid}
\usecolortheme{default}
\title{YOLOv8 Architecture}
\geometry{paperwidth=16cm,paperheight=10cm}
\author{Liu Yichao}
\date{}

\begin{document}

% Title slide
\begin{frame}
\titlepage
\end{frame}
\begin{frame}{Outline}
    \tableofcontents
\end{frame}
\begin{frame}{Mathematical Notations}
    \begin{columns}
        \begin{column}{0.5\textwidth}
            \begin{itemize}
                \item $X$: Input feature maps
                \item $X_{first}$, $X_{second}$: Split portions of input
                \item $F_i$: $i$-th convolutional block
                \item $S \times S$: Detection grid size
                \item $p_c$: Confidence score
                \item $(b_x, b_y)$: Center coordinates
                \item $(b_w, b_h)$: Width and height
            \end{itemize}
        \end{column}
        \begin{column}{0.5\textwidth}
            \begin{itemize}
                \item $W$, $b$: Learnable parameters
                \item $\sigma$: Sigmoid activation
                \item $L_{cls}$, $L_{box}$, $L_{dfl}$: Loss components
                \item IoU: Intersection over Union
                \item $\rho$: Euclidean distance between centers
                \item $T$: IoU threshold for NMS
                \item $d$: Local detection density
            \end{itemize}
        \end{column}
    \end{columns}
\end{frame}


\section{C2f Backbone Module}
\begin{frame}{C2f Backbone Module}
    \begin{itemize}
        \item Replaces traditional CSP blocks with more efficient C2f
        \item Uses dense connections for better gradient flow
        \item Reduces computational complexity by ~15\%
        \item Mathematical form:
            \begin{equation}
            C2f(X) = Concat[X_{first}, F_n(F_{n-1}(...F_1(X_{second})))]
            \end{equation}
        \item $X_{first}$, $X_{second}$: Split portions of input
        \item $F_i$: $i$-th convolutional block
    \end{itemize}
\end{frame}

\section{Anchor-Free Detection}
\begin{frame}{Anchor-Free Detection Head}
    \begin{itemize}
        \item Fully anchor-free approach
        \item Each grid cell directly predicts:
            \begin{equation}
            \hat{y} = \{p_c, b_x, b_y, b_w, b_h, p_1, p_2, ..., p_C\}
            \end{equation}
        \item No predefined anchor boxes
        \item More flexible object localization
        \item $p_c$: Confidence score
        \item $(b_x, b_y)$: Center coordinates
        \item $(b_w, b_h)$: Width and height
    \end{itemize}
\end{frame}

\section{Decoupled Head}
\begin{frame}{Decoupled Head Architecture}
    \begin{itemize}
        \item Separate branches for:
            \begin{align}
            \text{Classification: }C(F) &= \sigma(W_c \cdot F + b_c) \\
            \text{Regression: }R(F) &= W_r \cdot F + b_r \\
            \text{Confidence: }O(F) &= \sigma(W_o \cdot F + b_o)
            \end{align}
        \item Independent optimization for each task
        \item Reduces training conflicts
        \item $W$, $b$: Learnable parameters
        \item $\sigma$: Sigmoid activation
    \end{itemize}
\end{frame}

\section{Loss Functions}
\begin{frame}{Loss Function Formulation}
    Composite loss:
    \begin{equation}
    L_{total} = \lambda_{cls}L_{cls} + \lambda_{box}L_{box} + \lambda_{dfl}L_{dfl}
    \end{equation}

    \begin{itemize}
        \item $L_{cls}$: Binary Cross-Entropy for classification
        \item $L_{box}$: CIoU loss for bounding box regression
        \item $L_{dfl}$: Distribution Focal Loss for coordinate modeling
    \end{itemize}
\end{frame}

\section{Dynamic NMS}
\begin{frame}{Dynamic NMS Implementation}
    \begin{itemize}
        \item Adaptively adjusts IoU threshold based on detection density
        \begin{equation}
        T = T_{base} \cdot \exp(-\beta \cdot d)
        \end{equation}
        \item Better performance in crowded scenes
        \item Allows more detections in dense regions
        \item Maintains strict filtering in sparse areas
        \item $\beta$: Hyperparameter for adjustment
        \item $T$: IoU threshold for NMS
        \item $d$: Local detection density
    \end{itemize}
\end{frame}

\begin{frame}{Results}
    \begin{itemize}
        \item 53.9 AP on COCO dataset
        \item Inference time: 0.39 ms on modern GPUs
        \item Significant advancement in speed-accuracy trade-off
        \item State-of-the-art for real-time object detection
    \end{itemize}
\end{frame}

\end{document}

% C2f Backbone Mechanism
% The C2f module rethinks feature propagation through dense cross-stage connections. Traditional CSP blocks process features through a sequence of convolutions with limited skip connections, creating potential gradient decay in deep layers. C2f addresses this by introducing bottleneck-preserved concatenation - maintaining direct access to earlier features while progressively refining them through bottleneck layers.
% This design serves two critical functions:
% Gradient Flow Enhancement: By preserving multiple pathways from initial to final features, backpropagated gradients encounter fewer vanishing risks, particularly beneficial in deep networks.
% Feature Diversity Maintenance: Concatenation of minimally processed early features with deeply transformed ones creates a spectrum of feature abstraction levels. This proves particularly effective for multi-scale object detection where both fine details (edges, textures) and high-level semantics (object parts) are essential.
% Anchor-Free Detection Rationale
% YOLOv8 transitions to anchor-free prediction by directly regressing box coordinates from grid cells. This eliminates several inherent constraints of anchor-based approaches:
% Scale Sensitivity: Anchors require careful pre-definition of aspect ratios and sizes, creating bias towards dataset-specific object dimensions. Direct regression adapts organically to object statistics.
% Spatial Misalignment: Anchors create a spatial offset that must be learned through box adjustments. The anchor-free approach models absolute position prediction, reducing translation estimation errors.
% The mechanism works by treating each grid cell as potential object center, predicting distance to four boundaries. This center-based prior aligns well with modern datasets where objects tend to occupy central regions of their bounding boxes.
% Decoupled Head Architecture
% Joint optimization of classification and regression tasks creates conflicting gradient signals - improved localization may come at the cost of classification accuracy and vice versa. YOLOv8's decoupled head introduces task-specific feature transformations:
% Classification Path: Emphasizes spatial invariance through deeper fully-connected layers, better capturing categorical features
% Regression Path: Prioritizes spatial precision with shallower convolutions, preserving localization-sensitive features
% The architectural separation allows each branch to develop specialized feature representations. Experimental evidence shows classification features become more rotation-invariant while regression features maintain strict spatial relationships.
% Optimized Loss Formulation
% Three key loss components address different aspects of detection quality:
% Distribution Focal Loss (DFL): Models bounding box coordinates as probability distributions rather than deterministic values. By learning the shape of possible locations, DFL better handles ambiguous object boundaries common in occluded or partially visible targets.
% CIoU Loss: Extends standard IoU by incorporating:
% Center distance penalty (prevents drifting predictions)
% Aspect ratio consistency term (preserves object shape priors)
% This geometric-aware loss proves particularly effective for elongated or irregularly shaped objects.
% Task-Balanced Weights: Dynamic adjustment of classification vs regression loss weights prevents either task from dominating optimization. The balance factor adapts based on training stage progress.
% Dynamic NMS Strategy
% Traditional NMS uses fixed IoU thresholds, creating a fundamental trade-off between recall in crowded scenes and precision in sparse regions. YOLOv8's adaptive thresholding follows two principles:
% Density Awareness: Local object density estimated through kernel density estimation guides threshold relaxation in crowded areas
% Confidence Compensation: High-confidence predictions receive more lenient suppression thresholds, preserving correctly detected overlapping objects
% The mechanism implements this through a differentiable adjustment layer that modulates NMS thresholds based on spatial context, effectively creating "soft" suppression regions rather than binary keep/discard decisions.
% System-Level Synergy
% These components interact through several feedback loops:
% The C2f backbone's rich features enable accurate initial predictions for the anchor-free head
% Decoupled heads produce cleaner gradients that backpropagate more effectively through dense connections
% Adaptive NMS compensates for remaining localization errors in post-processing
% The unified design demonstrates how coordinated architectural improvements can compound detection accuracy without proportional computational cost increases. Each mechanism addresses specific limitations in previous detection pipelines while maintaining compatibility with real-time constraints.
